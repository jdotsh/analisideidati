\documentclass[11pt]{article}
\usepackage[italian]{babel}
\usepackage[utf8]{inputenc}
\usepackage{graphicx}
\usepackage{float}
\usepackage[normalem]{ulem}
\newcommand{\numpy}{{\tt numpy}}    % tt font for numpy

\topmargin -.5in
\textheight 9in
\oddsidemargin -.25in
\evensidemargin -.25in
\textwidth 7in

\begin{document}

% ========== Edit your name here
\author{What are the odds?}
\title{Manuale dell'ingegnere intrippato con la statistica}

\maketitle

\medskip
\section{Statistica descrittiva}
\subsection{Le grandezze che sintetizzano i dati}
\subsubsection{Media}
Dato un insieme $x_1,x_2,...,x_n$ di dati, si dice media campionaria la media aritmetica di questi valori.
\begin{displaymath}
\overline{x}:=\frac{1}{n}\sum_{i=1}^{n}x_i
\end{displaymath}
\subsubsection{Mediana}
Dato un insieme di dati di ampiezza $n$, lo si ordini dal minore al maggiore. La mediana è il valore che occupa la posizioone $\frac{n+1}{2}$ in caso di un insieme dispari, o la media tra $\frac{n}{2}$ e $\frac{n}{2}+1$ se pari.
\subsubsection{Moda}
La moda campionaria di un insieme di dati, se esiste, è l'unico valore che ha frequenza massima.
\subsubsection{Varianza e deviazione standard campionarie}
Dato un insieme di dati $x_1,x_2,...,x_n$, si dice varianza campionaria ($s^2$), la quantità
\begin{displaymath}
s^2:=\frac{1}{n-1}\sum_{i=1}^{n}(x_i-\overline{x})^2
\end{displaymath}
Una comodità per il calcolo è che 
\begin{displaymath}
\sum_{i=1}^{n}(x_i-\overline{x})^2 = \sum_{i=1}^{n}x_i^2-n\overline{x}^2
\end{displaymath}
Si dice \textbf{deviazione standard campionaria} e si denota con $s$, la quantità
\begin{displaymath}
s:=\sqrt{\frac{1}{n-1}\sum_{i=1}^{n}(x_i-\overline{x})^2}
\end{displaymath}
\begin{center}
    (la radice quadrata di $s^2$)
\end{center}
\subsubsection{Percentili campionari e box plot}
Sia k un numero intero $0\le k \le 100$. Dato un campione di dati, esiste sempre un dato che è contemporaneamente maggiore del $k$ percento dei dati, e minore del $100-k$ percento. Per trovare questo dato, dati $n$ e $p=\frac{k}{100}$:
\begin{enumerate}
    \item Disponiamo i dati in ordine crescente
    \item Calcoliamo $np$
    \item Il numero cercato è quello in posizione $np$, arrotondato per eccesso se non intero.
\end{enumerate}
Il 25-esimo percentile si dice \textit{primo quartile}, il 50-esimo \textit{secondo} (ed è pari alla mediana), il 75-esimo \textit{terzo}. Il box plot è un grafica con un quadrato sulla linea dei dati, con i lati sul primo e terzo quartile, e un segno sul secondo.
\subsection{Disuguaglianza di Chebyshev}
Siano $\overline{x}$ e $s$ media e deviazione standard campionarie di un insieme di dati. Nell'ipotesi che $s>0$, la disuguaglianza di Chebyshev afferma che per ogni reale $k\ge 1$, almeno una frazione $(1-1/k^2)$ dei dati cade nell'intervallo che va da $\overline{x}-ks$ a $\overline{x}+ks$.
Usando il \sout{pessimo} \textit{fantastico} linguaggio da statista: sia assegnato un insieme di dati $x_1,...,x_n$ con media campionaria $\overline{x}$ e deviazione standard campionaria $s>0$. Denotiamo con $S_k$ l'insieme degli indici corrispondenti a dati compresi tra $\overline{x}-ks$ e $\overline{x}+ks$. Sia $\#S_k$ il numero dei suddetti. Allora abbiamo che
\begin{displaymath}
\frac{\#S_k}{n}\ge 1 - \frac{n-1}{nk^2} >1-\frac{1}{k^2}
\end{displaymath}
\subsection{Insiemi di dati bivariati e coefficiente di correlazione campionaria}
A volte non abbiamo a che fare con dati singoli, ma con coppie di numeri, tra i quali sospettiamo l'esistenza di relazioni. Dati di questa forma prendono il nome di \textit{campione bivariato}. Uno strumento utile è il diagramma di dispersione. Una questione interessante è capire se vi sia correlazione tra i dati accoppiati. Parleremo di correlazione positiva quando abbiamo una proporzionalità diretta tra i due, di correlazione negativa quando abbiamo una proporzionalità inversa.
\subsubsection{Coefficiente di correlazione campionaria}
Dato un campione bivariato $(x_i,y_i)$, sono definite le medie $\overline{x}$ e $\overline{y}$. Possiamo senz'altro dire che se un valore $x_i$ è grande rispetto alla media, la differenza $x_i-\overline{x}$ sarà positiva, mentre se $x_i$ è piccolo, la differenza sarà negativa. Quindi, considerando il prodotto $(x_i-\overline{x})(y_i-\overline{y})$, sarà positivo per correlazioni positive, negativo per correlazioni negative. Se l'intero campione mostra quindi un'elevata correlazione, ci aspettiamo che la somma di tutti i prodotti $\sum_{i=1}^n(x_i-\overline{x})(y_i-\overline{y})$ darà una buona stima della correlazione. Normalizziamola dividendo per $(n-1)$ e per il prodotto delle deviazione standard campionarie, e otteniamo il \textbf{coefficiente di correlazione campionaria}
\begin{displaymath}
r:=\frac{\sum_{i=1}^{n}(x_i-\overline{x})(y_i-\overline{y})}{(n-1)s_x s_y}
\end{displaymath}
con $s_x$ e $s_y$ deviazioni standard campionarie di $x$ e $y$.
\subsubsection{Proprietà del coefficiente di correlazione campionaria}
Sebbene parleremo meglio di questo bastardo nella sezione sulla regressione, elenchiamo qui alcune proprietà:
\begin{enumerate}
    \item $-1 \le r \le 1$
    \item Se per opportune costanti a e b, con $b>0$ sussiste la relazione lineare $y_i = a+b_x$, allora $r=1$.
    \item Se per opportune costanti a e b, con $b<0$ sussiste la relazione lineare $y_i = a+b_x$, allora $r=-1$.
    \item Se $r$ è il coefficiente di correlazione del campione $(x_i, y_i)$, $i=1,...,n$, allora lo è anche per il campione $(a+bx_i, c+dy_i)$, purché le costanti $a$ e $b$ abbiano lo stesso segno.
\end{enumerate}
\end{document}
