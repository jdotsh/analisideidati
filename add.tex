\documentclass[11pt]{article}
\usepackage[italian]{babel}
\usepackage[utf8]{inputenc}
\usepackage{graphicx}
\usepackage{float}
\usepackage{amsmath}
\usepackage{amsfonts}
\usepackage[normalem]{ulem}
\newcommand{\numpy}{{\tt numpy}}    % tt font for numpy

\topmargin -.5in
\textheight 9in
\oddsidemargin -.25in
\evensidemargin -.25in
\textwidth 7in

\begin{document}

% ========== Edit your name here
\author{What are the odds?}
\title{Manuale dell'ingegnere intrippato con la statistica}

\maketitle

\medskip
\section{Statistica descrittiva}
\subsection{Le grandezze che sintetizzano i dati}
\subsubsection{Media}
Dato un insieme $x_1,x_2,...,x_n$ di dati, si dice media campionaria la media aritmetica di questi valori.
\begin{displaymath}
\overline{x}:=\frac{1}{n}\sum_{i=1}^{n}x_i
\end{displaymath}
\subsubsection{Mediana}
Dato un insieme di dati di ampiezza $n$, lo si ordini dal minore al maggiore. La mediana è il valore che occupa la posizioone $\frac{n+1}{2}$ in caso di un insieme dispari, o la media tra $\frac{n}{2}$ e $\frac{n}{2}+1$ se pari.
\subsubsection{Moda}
La moda campionaria di un insieme di dati, se esiste, è l'unico valore che ha frequenza massima.
\subsubsection{Varianza e deviazione standard campionarie}
Dato un insieme di dati $x_1,x_2,...,x_n$, si dice varianza campionaria ($s^2$), la quantità
\begin{displaymath}
s^2:=\frac{1}{n-1}\sum_{i=1}^{n}(x_i-\overline{x})^2
\end{displaymath}
Una comodità per il calcolo è che 
\begin{displaymath}
\sum_{i=1}^{n}(x_i-\overline{x})^2 = \sum_{i=1}^{n}x_i^2-n\overline{x}^2
\end{displaymath}
Si dice \textbf{deviazione standard campionaria} e si denota con $s$, la quantità
\begin{displaymath}
s:=\sqrt{\frac{1}{n-1}\sum_{i=1}^{n}(x_i-\overline{x})^2}
\end{displaymath}
\begin{center}
    (la radice quadrata di $s^2$)
\end{center}
\subsubsection{Percentili campionari e box plot}
Sia k un numero intero $0\le k \le 100$. Dato un campione di dati, esiste sempre un dato che è contemporaneamente maggiore del $k$ percento dei dati, e minore del $100-k$ percento. Per trovare questo dato, dati $n$ e $p=\frac{k}{100}$:
\begin{enumerate}
    \item Disponiamo i dati in ordine crescente
    \item Calcoliamo $np$
    \item Il numero cercato è quello in posizione $np$, arrotondato per eccesso se non intero.
\end{enumerate}
Il 25-esimo percentile si dice \textit{primo quartile}, il 50-esimo \textit{secondo} (ed è pari alla mediana), il 75-esimo \textit{terzo}. Il box plot è un grafica con un quadrato sulla linea dei dati, con i lati sul primo e terzo quartile, e un segno sul secondo.
\subsection{Disuguaglianza di Chebyshev}
Siano $\overline{x}$ e $s$ media e deviazione standard campionarie di un insieme di dati. Nell'ipotesi che $s>0$, la disuguaglianza di Chebyshev afferma che per ogni reale $k\ge 1$, almeno una frazione $(1-1/k^2)$ dei dati cade nell'intervallo che va da $\overline{x}-ks$ a $\overline{x}+ks$.
Usando il \sout{pessimo} \textit{fantastico} linguaggio da statista: sia assegnato un insieme di dati $x_1,...,x_n$ con media campionaria $\overline{x}$ e deviazione standard campionaria $s>0$. Denotiamo con $S_k$ l'insieme degli indici corrispondenti a dati compresi tra $\overline{x}-ks$ e $\overline{x}+ks$. Sia $\#S_k$ il numero dei suddetti. Allora abbiamo che
\begin{displaymath}
\frac{\#S_k}{n}\ge 1 - \frac{n-1}{nk^2} >1-\frac{1}{k^2}
\end{displaymath}
\subsection{Insiemi di dati bivariati e coefficiente di correlazione campionaria}
A volte non abbiamo a che fare con dati singoli, ma con coppie di numeri, tra i quali sospettiamo l'esistenza di relazioni. Dati di questa forma prendono il nome di \textit{campione bivariato}. Uno strumento utile è il diagramma di dispersione. Una questione interessante è capire se vi sia correlazione tra i dati accoppiati. Parleremo di correlazione positiva quando abbiamo una proporzionalità diretta tra i due, di correlazione negativa quando abbiamo una proporzionalità inversa.
\subsubsection{Coefficiente di correlazione campionaria}
Dato un campione bivariato $(x_i,y_i)$, sono definite le medie $\overline{x}$ e $\overline{y}$. Possiamo senz'altro dire che se un valore $x_i$ è grande rispetto alla media, la differenza $x_i-\overline{x}$ sarà positiva, mentre se $x_i$ è piccolo, la differenza sarà negativa. Quindi, considerando il prodotto $(x_i-\overline{x})(y_i-\overline{y})$, sarà positivo per correlazioni positive, negativo per correlazioni negative. Se l'intero campione mostra quindi un'elevata correlazione, ci aspettiamo che la somma di tutti i prodotti $\sum_{i=1}^n(x_i-\overline{x})(y_i-\overline{y})$ darà una buona stima della correlazione. Normalizziamola dividendo per $(n-1)$ e per il prodotto delle deviazione standard campionarie, e otteniamo il \textbf{coefficiente di correlazione campionaria}
\begin{displaymath}
r:=\frac{\sum_{i=1}^{n}(x_i-\overline{x})(y_i-\overline{y})}{(n-1)s_x s_y}
\end{displaymath}
con $s_x$ e $s_y$ deviazioni standard campionarie di $x$ e $y$.
\subsubsection{Proprietà del coefficiente di correlazione campionaria}
Sebbene parleremo meglio di questo bastardo nella sezione sulla regressione, elenchiamo qui alcune proprietà:
\begin{enumerate}
    \item $-1 \le r \le 1$
    \item Se per opportune costanti a e b, con $b>0$ sussiste la relazione lineare $y_i = a+b_x$, allora $r=1$.
    \item Se per opportune costanti a e b, con $b<0$ sussiste la relazione lineare $y_i = a+b_x$, allora $r=-1$.
    \item Se $r$ è il coefficiente di correlazione del campione $(x_i, y_i)$, $i=1,...,n$, allora lo è anche per il campione $(a+bx_i, c+dy_i)$, purché le costanti $a$ e $b$ abbiano lo stesso segno.
\end{enumerate}
\section{Elementi di probabilità}
\subsection{Spazio degli esiti ed eventi}
Si dice spazio degli esiti l'insieme di tutti gli esiti possibili di un esperimento. Se ad esempio l'esito dell'esperimento fosse il sesso di un neonato, lo spazio degli esiti sarebbe 
\begin{displaymath}
S=\{f,m\}
\end{displaymath}
I sottoinsiemi dello spazio degli esiti si dicono \textbf{eventi}, quindi un evento E è un insieme i cui elementi sono esiti possibili. Si dice $E^c$ l'opposto dell'evento, quindi $P(E^c) = 1 - P(E)$. Risulta ovvio che $1 = P(E^c) + P(E)$.
Se abbiamo due eventi qualsiasi, la loro unione $P(E\cup F) = P(E) + P(F) - P(E\cap F)$.
\subsection{Spazi di esiti equiprobabili}
Per tanti esperimenti è naturale assumere che ognuno degli esiti abbia la stessa probabilità di accadere. Abbiamo quindi che la probabilità che $E$ accada è pari a $P(E)=\frac{1}{N}$.
\subsubsection{Principio di enumerazione}
Consideriamo la realizzazione di due diversi esperimenti che possono avere rispettivamente $m$ ed $n$ esiti. Allora complessivamente avremo $mn$ risultati.
\subsection{Coefficiente binomiale}
Vogliamo ora determinare il numero di diversi gruppi di $r$ oggetti che si possono formare scegliendoli da un insieme di $n$. Ad esempio, quanti gruppi di 3 lettere possono formarsi dal gruppo \{A,B,C,D,E\}. In generale, poiché il numero di modi diversi di scegliere $r$ oggetti su $n$ tenendo conto dell'ordine è dato da $n(n-1)...(n-r+1)$, e poiché ogni gruppo di lettere viene contato $r!$ volte (uno per permutazione), il numero di gruppi di $r$ elementi su n totali è dato da
\begin{displaymath}
    \frac{n(n-1)...(n-r+1)}{r!} = \frac{n!}{r!(n-r!)} = {n\choose r}
\end{displaymath}
\subsection{Probabilità condizionata}
Vogliamo ora calcolare la probabilità che un evento accada, appurato che ne è accaduto un altro. Ad esempio, lanciamo due dadi. L'evento E cercato è che il risultato sia 8. L'evento F già accaduto è che il primo dato risulta in un 3. Si dice probabilità condizionata di E dato F
\begin{displaymath}
    P(E|F) = \frac{P(E\cap F)}{P(F)}
\end{displaymath}
\subsection{Fattorizzazione di un evento e formula di Bayes}
Siano E ed F due eventi qualsiasi. È possibile esprimere E come
\begin{displaymath}
    P(E) = P(E\cap F) + P(E\cap F^c)
\end{displaymath}
Visto inoltre che i due sono eventi disgiunti, si ha che
\begin{gather*}
    P(E) = P(E|F)P(F) + P(E|F^c)P(F^c)\\
    P(E|F)P(F) + P(E|F^c)[1-P(F)]
\end{gather*}
Questa \textit{orribile} equazione, ci mostra che la probabilità dell'evento E si può ricavare come media pesata delle probabilità condizionali di E sapendo che: F si è verificato e non si è verificato. I pesi sono ovviamente le probabilità degli eventi a cui si condiziona.
\subsection{Eventi indipendenti}
Due eventi si dicono indipendenti quando il risultato di uno non influenza l'altro. In altre parole, significa che avendo due eventi E ed F, se so che F è accaduto, la probabilità che accada E non cambia.
\begin{displaymath}
    P(E\cap F) = P(E) P(F)
\end{displaymath}
\section{Variabili aleatorie e valore atteso}
Quando realizziamo un esperimento casuale, non sempre siamo interessati a tutti i risultati del suddetto. Se ad esempio lanciassimo due dadi, potrebbe interessarci la sola somma e non i singoli risultati. 
Queste quantità di interesse sono dette \textbf{variabili aleatorie}. Siccome il valore di questa variabile è dato dal risultato dell'esperimento, possiamo assegnare delle probabilità a queste. Queste variabili aleatorie hanno una \textit{funzione indicatrice} definita, ad esempio, così:
\begin{displaymath}
    I:=
    \begin{cases}
        1 \mbox{ se } X = 1 \mbox{ o } 2\\
        0 \mbox{ se } X = 0
    \end{cases}
\end{displaymath}
Variabili aleatorie con un numero finito o numerabile di valori possibili sono dette \textbf{discrete}. Esistono anche variabili aleatorie \textbf{continue}.
\subsubsection{Funzione di ripartizione}
La funzione di ripartizione F di una variabile aleatoria X, è definita, per ogni numero reale $x$, tramite
\begin{displaymath}
    F(x) := P(X\le x)
\end{displaymath}
Quindi $F(x)$ esprime la probabilità che la variabile aleatoria X assuma un valore \textit{minore o uguale} a $x$. Tutte le questioni di probabilità che si possano sollevare su una variabile aleatoria, ammettono una risposta in termini della sua funzione di ripartizione.
\subsection{Variabili aleatorie discrete e continue}
Se X è una variabile aleatoria discreta, la sua funzione di massa di probabilità, o funzione di massa, si definisce nel modo seguente:
\begin{displaymath}
    p(a) :=P(X=a)
\end{displaymath}
La funzione $p(a)$ è non nulla su un insieme al più numerabile di valori. Infatti, se $x_1,x_2,...,x_n$ sono i possibili valori di X, allora
\begin{gather*}
    p(x_i) > 0 \hspace{10px}i=1,2,...\\
    p(x) = 0 \hspace{10px}\mbox{tutti gli altri valori di x}
\end{gather*}
Siccome X deve assumere i suddetti valori, necessariamente deve essere vero che
\begin{displaymath}
    \sum_{i=1}^\infty p(x_i) = 1
\end{displaymath}
Una variabile aleatoria che possa assumere un'infinità non numerabile di valori, non potrà essere discreta. Si dirà \textbf{continua} se esiste una funzione non negativa $f$, definita su tutto $\mathbb{R}$, avente la proprietà che per ogni insieme B di numeri reali,
\begin{displaymath}
    P(X\in B) = \int_B f(x) dx
\end{displaymath}
Questa funzione è detta \textbf{funzione di densità di probabilità}. L'equazione dice che la probabilità che una variabile aleatoria continua X appartenga a un insieme B si può trovare integrando la sua densità su tale insieme. Pare ovvio che
\begin{displaymath}
    1=P(X\in \mathbb{R}) = \int_{-\infty}^{\infty} f(x)dx
\end{displaymath}
Tutte le probabilità che riguardano una variabile aleatoria continua possono essere espresse in funzione della sua densità di probabilità:
\begin{displaymath}
    P(a\le X \le b) = \int_a^b f(x)dx
\end{displaymath}
Se poniamo $b=a$, notiamo che la probabilità che una variabile aleatoria continua assuma un valore particolare $a$ è nulla:
\begin{displaymath}
    P(X=a) = \int_a^a f(x)dx = 0
\end{displaymath}
Leghiamo la funzione di ripartizione F alla densità $f$ così:
\begin{displaymath}
    F(a) := P(X \in (-\infty,a]) = \int_{-\infty}^a f(x)dx
\end{displaymath}
Derivando entrambi otteniamo la relazione fondamentale:
\begin{displaymath}
    \frac{d}{da}F(a) = f(a)
\end{displaymath}
La densità è quindi la derivata della funzione di ripartizione. 
Notiamo che quando conosciamo la funzione di massa di probabilità di una variabile aleatoria discreta, o la funzione di densità di probabilità di una continua, abbiamo abbastanza informazioni per poter calcolare le probabilità di ogni evento che dipenda dalla sola variabile aleatoria. 
\subsection{Coppie e vettori di variabili aleatorie}
Ci sono situazioni in cui abbiamo necessità di studiare le \textbf{relazioni} tra variabili aleatorie multiple. Per specificare la relazione tra due variabili aleatorie X e Y, il primo passo è estendere il concetto di funzione di ripartizione.
Siano quindi X e Y due variabili aleatorie che riguardano lo stesso esperimento casuale. Si dice \textit{funzione di ripartizione congiunta} di X e Y la funzione di due variabili seguente:
\begin{displaymath}
    F(x,y) := P(X\le x, Y \le y)
\end{displaymath}
dove la virgola denota l'intersezione tra gli eventi.
La conoscenza di questa funzione permette, almeno in teoria, di calcolare le probabilità di tutti gli eventi che dipendono, singolarmente o congiuntamente, da X e Y. 
\subsubsection{Distribuzione congiunta per variabili aleatorie discrete}
Se sappiamo che un vettore aleatorio è di tipo discreto, possiamo definire e utilizzare la funzione di massa di probabilità.
Se X e Y sono variabili aleatorie discrete che assumono i valori $x_1, x_2,...$ e $y_1,y_2,...$, la funzione
\begin{displaymath}
    p(x_i, y_j) := P(X=x_i, Y=y_j),\hspace{10px}i=1,2,...\hspace{5px}j=1,2,...
\end{displaymath}
è la loro funzione di massa di probabilità congiunta.
Le funzioni di massa individuali si possono ricavare da questa, notando che, siccome Y deve assumere uno dei valori $y_j$, l'evento $\{X=x_i\}$, può essere visto come l'unione al variare di $j$ degli eventi $\{X=x_i, Y=y_j\}$, che sono mutuamente esclusivi. Da qui:
\begin{displaymath}
    p_X(x_i) := P(X=x_i) = \sum_j p(x_i, y_j)
\end{displaymath}
Anche se le individuali possono essere ricavate dalla congiunta, la congiunta non può essere ricavata dalle condizionali. 
\subsubsection{Distribuzione congiunta per variabili aleatorie continue}
Due variabili aleatorie X e Y sono congiuntamente continue se esiste una funzione non negativa $f(x,y)$ tale che, per ogni sottoinsieme C del piano cartesiano,
\begin{displaymath}
    P((X,Y)\in C) = \int\int_{(x,y)\in C} f(x,y)dxdy
\end{displaymath}
questa è detta \textbf{densità congiunta} delle variabili aleatorie X e Y.
Otteniamo inoltre che
\begin{displaymath}
    P(X \in A, Y \in B) = \int_B\int_A f(x,y)dxdy
\end{displaymath} 
E, in conclusione,
\begin{displaymath}
    P(X \in A) = \int_A f_X(x) dx
\end{displaymath}
Per ricavare le individuali, otteniamo
\begin{gather*}
    f_X(x) =\int_{-\infty}^\infty f(x,y) dy\\
    f_Y(y) =\int_{-\infty}^\infty f(x,y) dx
\end{gather*}
\subsubsection{Variabili aleatorie indipendenti}
Due variabili aleatorie sono indipendenti se tutti gli eventi relativi alla prima sono indipendenti da tutti quelli relativi alla seconda. 
La definizione è che se, per ogni coppia di insiemi di numeri reali A e B è soddisfatta
\begin{displaymath}
    P(X \in A, Y \in B) = P(X\in A)P(Y \in B)
\end{displaymath}
le due V.A. sono indipendenti. 
Se le V.A. sono discrete, l'equazione equivale a dire che la funzione di massa congiunta è il prodotto delle marginali:
\begin{displaymath}
    p(x,y) = p_X(x) p_Y(y)
\end{displaymath}
Possiamo generalizzare le osservazioni suddette anche per vettori di variabili aleatorie. \textit{Lo faremo? Non credo proprio.}
\subsubsection{Distribuzioni condizionali}
Le relazioni esistenti tra due variabili aleatorie possono essere chiarite dallo studio della distribuzione condizionale di una delle due, dato il valore dell'altra. Si ricorda che che presi comunque due eventi E e F con $P(F)>0$, la probabilità di E condizionata a F è data da
\begin{displaymath}
    P(E|F):=\frac{P(E\cap F)}{P(F)}
\end{displaymath}
è naturale applicare questo schema anche alle variabili aleatorie discrete.
Siano X e Y due variabili aleatorie discrete con funzione di massa congiunta $p(\cdot , \cdot)$, diciamo funzione di massa di probabilità condizionata di X dato Y e si indica con $p_{X|Y}(\cdot | \cdot)$, la funzione di due variabili così definita:
\begin{gather*}
    p_{X|Y}(x|y) := P(X=x|Y=y)\\ 
    \frac{p(x,y)}{p_Y(y)},\hspace{10px}\forall x \forall y \mbox{ con }p_Y(y)>0
\end{gather*}
Se $y$ non è un valore possibile di Y, ovvero se $P(Y=y)=0$, la quantità $p_{X|Y}(x|y)$ non è definita. 
\subsection{Valore atteso}
Uno dei concetti più importanti di tutta la teoria della probabilità \textit{(ziocane)} è quello di valore atteso. Esso è definito come il numero
\begin{displaymath}
    E[X]:=\sum_i x_i P(X=x_i)
\end{displaymath}
In altri termini, si tratta della media pesata dei valori possibili di X, usando come pesi le probabilità che vengano assunti. È ovvio che nel caso di V.A. continue, il giochino non funziona. Definiamo quindi il valore atteso di una V.A. continua con funzione di densità $f$, come
\begin{displaymath}
    E[X]:=\int_{-\infty}^\infty xf(x)dx
\end{displaymath}
\subsection{Proprietà del valore atteso}
Consideriamo una V.A. di cui conosciamo la distribuzione. \textit{What if, }se anziché calcolare il valore atteso di X, volessimo calcolare quello di una funzione $g(X)$? Notiamo che $g(X)$ è comunque una variabile aleatoria. Ricaviamo quindi la sua distribuzione, e ne calcoliamo il valore atteso. Ponendo le cose in maniera rigorosa:
\begin{gather*}
E[g(X)] = \sum_x g(x)p(x)\mbox{ per V.A. discrete}\\ 
E[g(X)] =\int_{-\infty}^\infty g(x)f(x)dx\mbox{ per V.A. continue}
\end{gather*}
Per ogni coppia di costanti reali $a$ e $b$, abbiamo anche che
\begin{displaymath}
    E[aX+b] = aE[X]+b\hspace{10px}\mbox{e quindi}\hspace{10px}E[aX]=aE[X]
\end{displaymath}
\subsubsection{Valore atteso della somma di variabili aleatorie}
Con \textit{complessi calcoli tendenzialmente inutili}, otteniamo che 
\begin{displaymath}
    E[X+Y] = E[X] + E[Y]
\end{displaymath}
Tale risultato vale sia nel caso discreto, che in quello continuo.
\subsection{Varianza}
A volte, conoscere la media di una distribuzione non basta. \textit{Se inserissimo l'autore di questi riassunti con la testa in un freezer e i piedi in un forno, la temperatura media sarebbe abbastanza ok, l'autore no.} Per questo, è utile conoscere quanto i valori si allontanano dalla media. Questo è proprio il compito della \textbf{varianza}. Sia X una variabile aleatoria con media $\mu$, la varianza di x, che denotiamo con $Var(X)$ è la quantità
\begin{displaymath}
    Var(X):=E[(X-\mu)^2]
\end{displaymath}
o, in alternativa (\textit{questa è molto più comoda})
\begin{displaymath}
    Var(X) = E[X^2]-E[X]^2
\end{displaymath}
\subsection{La covarianza e la varianza della somma di V.A.}
Come sappiamo, la media della somma di V.A. coincide con la somma delle loro medie. Per la varianza, in generale, questo non è vero. \textbf{In un caso sì: quando le V.A. sono indipendenti}. Prima di tutto, però, definiamo il concetto di Covarianza: date due V.A. X e Y di media $\mu_X$ e $\mu_Y$, essa vale
\begin{displaymath}
    Cov(X,Y) := E[(X-\mu_X)(Y-\mu_Y)]
\end{displaymath}
o, in alternativa
\begin{displaymath}
    Cov(X,Y) := E[XY] - E[X]E[Y]
\end{displaymath}
Derivano anche alcune semplici proprietà
\begin{gather*}
    Cov(X,Y)=Cov(Y,X)\\ 
    Cov(X,X)=Var(X)\\ 
    Cov(aX,Y)=aCov(X,Y)=Cov(X,aY)
\end{gather*}
E se avessimo 3 V.A.?
\begin{displaymath}
    Cov(X+Y,Z) = Cov(X,Z)+Cov(Y,Z)
\end{displaymath}
Inoltre, generalizzando i concetti, se avessimo n V.A. $X_1,...,X_n$ e n $Y_1,...,Y_n$
\begin{displaymath}
    Cov\left(\sum_{i=1}^nX_i,\sum_{j=1}^mY_j\right) = \sum_{i=1}^n\sum_{j=1}^m Cov(X_i,Y_j)
\end{displaymath}
\subsubsection{Variabili aleatorie indipendenti}
Se abbiamo due V.A. X e Y indipendenti, sappiamo che
\begin{displaymath}
    E[XY] = E[X]E[Y]
\end{displaymath}
Questo implica inoltre che
\begin{displaymath}
    Cov(X,Y) = 0
\end{displaymath}
e quindi, se abbiamo n V.A., la varianza della somma è la somma delle varianze.
\begin{displaymath}
    Var(\sum_{i=1}^n X_i) = \sum_{i=1}^n Var(X_i)
\end{displaymath}
\section{La funzione generatrice dei momenti}
La \textit{funzione generatrice dei momenti}, o più semplicemente, la funzione generatrice $\phi$ di una V.A. X, è definita, per tutti i $t$ reali per i quali il valore atteso di $e^{tX}$ ha senso, dall'espressione
\begin{displaymath}
    \phi(t):=E[e^{tX}]=
    \begin{cases}
        \sum_x e^{tx}p(x)\hspace{10px}\mbox{se X è discreta}\\ 
        \int_{-\infty}^\infty e^{tx} f(x)dx\hspace{10px}\mbox{se X è continua}
    \end{cases}
\end{displaymath} 
Il nome deriva dal fatto che tutti i momenti di cui è dotata X possono essere ottenuti derivando più volte nell'origine la funzione $\phi(t)$. Ad esempio,
\begin{displaymath}
    \phi'(t)=\frac{d}{dt}E[e^{tX}]=E[Xe^{tX}]
\end{displaymath}
Quindi, $\phi'(0) = E[X]$, e, più in generale, $\phi^n(0) = E[X^n]$.
Se X e Y sono variabili indipendenti con funzioni generatrici $\phi_X$ e $\phi_Y$, e se $\phi_{X+Y}$ è la funzione generatrice dei momenti di $X+Y$, allora
\begin{displaymath}
    \phi_{X+Y}(t) = \phi_X(t)\phi_Y(t)
\end{displaymath}
Un'osservazione interessante sulla generatrice dei momenti, è che essa \textit{determina la distribuzione}, ossia se due V.A. hanno identica generatrice, hanno identica legge(quindi funzione di ripartizione e funzione di massa).
\section{La legge debole dei grandi numeri}
Per introdurre la suddetta, prima enunciamo la \textbf{disuguaglianza di Markov}: se X è una variabile aleatoria che non è mai negativa. allora per ogni $a>0$
\begin{displaymath}
    P(X\ge a) \le \frac{E[X]}{a}
\end{displaymath}
Come corollario, ricaviamo la disuguaglianza di Chebyshev: data una V.A. X con media $\mu$ e varianza $\sigma^2$, allora per ogni $r>0$
\begin{displaymath}
    P(|X-\mu|\ge r) \le \frac{\sigma^2}{r^2}
\end{displaymath}
Otteniamo infine la \textbf{legge debole dei grandi numeri}. \textit{No, non quella con cui giustificate il vostro provarci con ogni essere vivente femminile.} Sia $X_1,X_2,...$ una successione di variabili aleatorie i.i.d\textit{(indipendenti, identicamente distribuite)}, tutte con media $\mu$. Allora, per ogni $\epsilon >0$:
\begin{displaymath}
    P\left(\left|\frac{X_1+...+X_n}{n} - \mu\right|>\epsilon \right)\rightarrow0\hspace{10px}\mbox{quando }n\rightarrow\infty
\end{displaymath}
\textit{Ora, vi chiederete: cosa me ne faccio? Posso spiegarla alle tipe in discoteca?\textbf{ No.}} Un'applicazione interessante è la seguente: supponiamo di ripetere in successione molte copie indipendenti di un esperimento, in ciascuna delle quali può verificarsi un certo evento E:
\begin{displaymath}
    X_i:=
    \begin{cases}
        1\mbox{ se E si realizza nell'esperimento i-esimo}\\ 
        0\mbox{ se E non si realizza}
    \end{cases}
\end{displaymath}
la sommatoria $X_1+...+X_n$ rappresenta il numero di prove - tra le prime n - in cui si è verificato l'evento E. Poiché 
\begin{displaymath}
    E[X_i]=P(X_i=1)=P(E)
\end{displaymath}
si deduce che la frazione delle \textit{n} prove nelle quali si realizza E, tende alla probabilità $P(E)$.
\section{Modelli di variabili aleatorie}
\textit{Siamo giunti a una sezione tanto interessante, quanto fastidiosa: quella in cui dovete ricordare uno sh*t ton di roba}.
\subsection{Variabili aleatorie di Bernoulli e binomiali}
Supponiamo di fare un esperimento che ha solo due esiti, \textit{successo e fallimento}. Sappiamo che
\begin{gather*}
    P(X=0)=1-p\\ 
    P(X=1)=p
\end{gather*}
Una V.A. con funzione di massa di probabilità come questa, è detta \textbf{Bernoulliana}. Il suo valore atteso $E[X] = p$.
Supponiamo ora di realizzare n esperimenti, ciascuno dei quali è descritto da una Bernoulliana. Se X denota il numero totale di successi, X si dice V.A. binomiale di parametri $(n,p)$. La funzione di massa di probabilità è data da:
\begin{displaymath}
    P(X=i)={n \choose i} p^i(1-p)^{n-i}
\end{displaymath}
con il solito coefficiente binomiale
\begin{displaymath}
    {n \choose i} := \frac{n!}{i!(n-i)!}
\end{displaymath}
Si noti che la somma delle probabilità di tutti i valori possibili è ovviamente 1.
\begin{displaymath}
    \sum_iP(X=i)=[p+(1-p)]^n=1
\end{displaymath}
Osserviamo che se $X_1$ e $X_2$ sono binomiali di parametri $(n_1,p)$ e $(n_2,p)$ e sono indipendenti, allora la somma $X_1+X_2$ è binomiale di parametri $(n_1+n_2,p)$.
\subsubsection{Calcolo esplicito della distribuzione binomiale}
Supponiamo che X sia binomiale di parametri $(n,p)$. Per poter calcolare operativamente la funzione di ripartizione:
\begin{displaymath}
    P(X\le i) = \sum_{k=0}^i {n \choose i} p^k (1-p)^{n-k}
\end{displaymath}
o la funzione di massa:
\begin{displaymath}
    P(X=i) = {n \choose i} p^i(1-p)^{n-1}
\end{displaymath}
è molto utile sapere che 
\begin{displaymath}
    P(X=k+1)=\frac{p}{1-p}\frac{n-k}{k+1}P(X=k)
\end{displaymath}
\subsection{Variabili aleatorie di Poisson}
Le variabili aleatorie di Poisson assumono solo valori interi non negativi.
In altri termini, definiamo \textit{una variabile aleatoria X che assuma i valori 0,1,2...} \textbf{poissoniana} di parametro $\lambda$, $\lambda>0$, se la sua funzione di massa di probabilità è data da
\begin{displaymath}
    P(X=i) = \frac{\lambda^i}{i!}e^{-\lambda}
\end{displaymath}
Calcoliamo la generatrice dei momenti $E[e^{tX}] = e^{-\lambda}e^{\lambda e^t}$, per poi derivarla e calcolarla in $X=0$.
Troviamo quindi che 
\begin{gather*}
    E[X] = \phi'(0) = \lambda\\
    Var(X) = \phi''(0)-E[X]^2 = \lambda
\end{gather*}
Una caratteristica interessante della poissoniana è che può essere utilizzata come approssimazione di una binomiale di parametri $(n,p)$, quando $n$ è molto grande e $p$ molto piccolo. 
Ponendo $\lambda=np$:
\begin{displaymath}
    P(X=i) \approx \frac{\lambda^i}{i!}e^{-\lambda}
\end{displaymath}
In altri termini, il totale dei successi di un numero n molto elevato di ripetizioni dell'esperimento, è una variabile aleatorie di con distribuzione approsimativamente di Poisson, con media $\lambda=np$.
\subsubsection{Calcolo esplicito della distribuzione di Poisson}
Se X è una variabile aleatoria di Poisson di media $\lambda$, allora
\begin{displaymath}
    \frac{P(X=i+1)}{P(X=i)}=\frac{\lambda^{i+1}e^{-\lambda}}{(i+1)!}\frac{i!}{\lambda^i e^{-\lambda}}=\frac{\lambda}{i+1}
\end{displaymath}
\textit{A cosa serve sto pippozzo? }Possiamo usarla per calcolare, a partire da $P(X=0) = e^{-\lambda}$, tutti gli altri. Ad esempio:
\begin{gather*}
    P(X=1) = \lambda P(X=0)\\
    P(X=2) = \frac{\lambda}{2} P(X=1)
\end{gather*}
\subsection{Variabili aleatorie ipergeometriche}
Una scatola contiene N batterie accettabili, ed M difettose. Si estraggono senza rimessa $n$ batterie, dando pari probabilità a ciascuno degli ${{N+M} \choose n}$ sottoinsiemi possibili. Se denotiamo con X il numero di batterie accettabili estratte,
\begin{displaymath}
    P(X=i)=\frac{{N \choose i} {N \choose {n-i}}}{{{N+M} \choose n}}\hspace{10px}i=0,1,...,n
\end{displaymath}
Una V.A. con massa di probabilità data da questa equazione, è detta \textbf{variabile aleatoria ipergeometrica} di parametri N, M e $n$.
\subsection{Variabili aleatorie uniformi}
Una variabile aleatoria continua si dice uniforme sull'intervallo $[\alpha,\beta]$, se ha funzione di densità data da:
\begin{displaymath}
    f(x)=
    \begin{cases}
        \frac{1}{\beta-\alpha} \mbox{ se }\alpha\le x \le \beta\\
        0\hspace{10px}\mbox{altrimenti} 
    \end{cases}
\end{displaymath}
Si noti che soddisfa le condizioni per essere una densità di probabilità, in quanto
\begin{displaymath}
    \int_{-\infty}^\infty f(x)dx =\frac{1}{\beta-\alpha}\int_\alpha^\beta dx=1
\end{displaymath}
Per poter assumere la distribuzione uniforme, nella pratica, occorre che la V.A. abbia come valori possibili i punti di un intervallo limitato $[\alpha,\beta]$, inoltre si deve poter supporre che essa abbia le stesse probabilità di cadere vicino a un qualunque punto dell'intervallo.
La probabilità che una V.A. X, uniforme su $[\alpha,\beta]$ è pari al rapport tra le lunghezze dei due intervalli,. Infatti, se $[a,b]$ è contenuto in $[\alpha,\beta]$
\begin{displaymath}
    P(a<X<b)=\frac{1}{\beta-\alpha}\int_a^b dx = \frac{b-a}{\beta-\alpha}
\end{displaymath}
\subsection{Variabili aleatorie normali}
Questa è la sezione più applicabile alla vita reale. \textit{Quella che, a forza di studiare statistica, avete dimenticato.} 
Una variabile aleatoria X si dice \textbf{normale} oppure \textbf{gaussiana} di parametri $\mu,\sigma^2$ e si scrive $X \sim N(\mu,\sigma^2)$ se X ha funzione di densità data da
\begin{displaymath}
    f(x)= \frac{1}{\sqrt{2\pi}\sigma} exp\left\{-\frac{(x-\mu)^2}{2\sigma^2}\right\} \hspace{10px}\forall x \in \mathbb{R} 
\end{displaymath}
La densità normale è una curva a campana simmetrica rispetto all'asse $x=\mu$, dove ha il massimo pari a $(\sigma\sqrt{2\pi})^{-1} \approx  0.399/\sigma$.
Come sempre, cerchiamo valore atteso e varianza, calcolando la generatrice $\phi = E[e^{tX}]$, derivandola e calcolandola in $X=0$. Otteniamo 
\begin{gather*}
    E[X]=\phi'(0)=\mu\\ 
    Var(X) = E[X^2]- E[X]^2 = \sigma^2
\end{gather*}
Un risultato importante è che se X è una gaussiana e Y una trasformazione lineare di X, allora Y è a sua volta una gaussiana. In altri termini:
sia $X\sim N(\mu, \sigma^2)$ e sia $Y=\alpha X + \beta$, dove $\alpha$ e $\beta$ sono due costanti reali, con $\alpha\neq 0$. Allora Y è una normale con media $\alpha\mu + \beta$ e varianza $\alpha^2\sigma^2$.
Un corollario della suddetta permette di ottenere il fantomatico processo di normalizzazione: se $X\sim N(\mu,\sigma^2)$, allora
\begin{displaymath}
    Z:=\frac{X-\mu}{\sigma}
\end{displaymath}
è una V.A. normale con media 0 e varianza 1. Questa roba è così importante che ha un nome, \textbf{normale standard}, e un simbolo:
\begin{displaymath}
    \Phi(x) := \frac{1}{\sqrt{2\pi}}\int_{-\infty}^x e^{\frac{-y^2}{2}}dy\hspace{10px}\forall x \in \mathbb{R}
\end{displaymath}
Il fatto che $Z:=(X-\mu)/\sigma$ abbia distribuzione normale standard quando X è gaussiana di media $\mu$ e varianza $\sigma^2$, ci permette di esprimere le probabilità relative di X in termini di probabilità su Z. Ad esempio per trovare $P(X<b)$, notiamo che $X<b$ s.s.se 
\begin{displaymath}
    \frac{X-\mu}{\sigma} < \frac{b-\mu}{\sigma}
\end{displaymath}
così che
\begin{displaymath}
    P(X<b) = \Phi\left(\frac{b-\mu}{\sigma}\right)
\end{displaymath}
Analogamente, per ogni $a<b$, si ha che 
\begin{displaymath}
    P(a<X<b) := \Phi\left(\frac{b-\mu}{\sigma}\right)-\Phi\left(\frac{a-\mu}{\sigma}\right)
\end{displaymath}
\subsection{Variabili aleatorie esponenziali}
Una V.A. continua, la cui funzione di densità è data da 
\begin{displaymath}
    f(x)=\begin{cases}
        \lambda e^{-\lambda x} \hspace{10px}\mbox{se }x\ge 0\\ 
        0\hspace{10px}\mbox{se }x<0
    \end{cases}
\end{displaymath}
per un opportuno valore della costante $\lambda>0$, si dice \textbf{esponenziale} con parametro (\textit{o intensità}) $\lambda$.
La funzione di ripartizione di una tale variabile è data da 
\begin{displaymath}
    F(x) = P(X\le x) = 1-e^{-\lambda x}\hspace{10px} x\ge 0
\end{displaymath}
Nella pratica, la distribuzione esponenziale può rappresentare il tempo di attes aprima che si verifichi un certo evento casuale. Calcoliamo la generatrice dei momenti $E[e^{tX}] = \frac{\lambda}{\lambda-t}$ con $t<\lambda$. Derivando e calcolando in 0:
\begin{gather*}
    E[X]=\phi'(0)=\frac{1}{\lambda}\\ 
    Var(X) = E[X^2] - E[X]^2 = \frac{1}{\lambda^2}
\end{gather*}
La proprietà fondamentale dell'esponenziale è la sua \textbf{assenza di memoria}. \textit{Che cazzo vuol dire? Nemmeno io ho memoria ma non ho libri da 40 euro che parlano di me.} Immaginiamo che X rappresenti il tempo che passa prima che un oggeto si rompa, dato t il tempo per cui ha già funzionato. Chiaramente, la probabilità che l'oggetto duri ancora per un tempo $s$ è $P(X>s+t | X>t)$. Otterremo che la condizione $X>t$ non varia la probabilità, e non ci dà quindi "informazioni utili".
Un'altra informazione utile, è che se abbiamo n V.A. esponenziali $X_1,...,X_n$ indipendenti, di parametri $\lambda_1,...,\lambda_n$, allora la V.A. $Y:=min(X_1,...,X_n)$ è un'esponenziale di parametro $\lambda_Y=\sum_{i=1}^n \lambda_i$.
\subsubsection{Il processo di Poisson}
Consideriamo una serie di eventi istantanei che avvengono a intervalli di tempo casuali, e sia $N(t)$ il numero di quanti se ne sono verificati nell'intervallo $[0,t]$. $N(t)$ si dice \textit{processo di Poisson} di intensità $\lambda$, $\lambda>0$, se
\begin{itemize}
    \item $N(0) = 0$: \textit{questa stabilisce che si iniziano a contare gli eventi al tempo 0.}
    \item Il numero degli eventi che hanno luogo in intervalli di tempo disgiunti sono indipendenti
    \item La distribuzione del numero di eventi che si verifica in un dato intervallo di tempo dipende solo dalla lunghezza dell'intervallo, e non dalla sua posizione.
    \item $\lim_{h\to0}\frac{P(N(h)=1)}{h} = \lambda$
    \item $\lim_{h\to0}\frac{P(N(h)\ge2)}{h} = 0$
\end{itemize}
Le ultime due affermano che se si considera un intervallo di tempo di lunghezza $h$, vi è approssimativamente possibilità $\lambda h$ che vi occorrà un evento solo, e circa una probabilità nulla che se ne verifichino due o più.
Se $N(t)$ è un processo di Poisson di intensità $\lambda$, allora 
\begin{displaymath}
    P(N(t)=k) = \frac{(\lambda t)^k}{k!} e^{-\lambda t}
\end{displaymath}
Ovvero, il numero di eventi in $[0,t]$ ha distribuzione di Poisson di media $\lambda t$. Inoltre, i tempi che separano gli eventi di un processo di Poisson, sono una successione di V.A. esponenziali di intensità $\lambda$.
\subsection{Variabili aleatorie di tipo gamma}
Una variabile aleatoria continua si dice avere distribuzione di tipo gamma di parametri $(\alpha,\lambda)$, con $\alpha>0, \lambda>0$ se la sua funzione di densità di probabilità è data da 
\begin{displaymath}
    f(x) =
    \begin{cases}
        \frac{\lambda^\alpha}{\Gamma(\alpha)}x^{\alpha-1}e^{-\lambda x} \hspace{10px}\mbox{se } x>0\\ 
        0  \hspace{10px}\mbox{se } x\le0
    \end{cases}
\end{displaymath}
dove $\Gamma$ denota la funzione gamma di Eulero, che è definita in modo da normalizzare l'integrale di $f$.
Per proprietà che ci asterremo dall'enunciare, possiamo calcolare la gamma sugli interi così:
\begin{displaymath}
    \Gamma(n) = (n-1)!
\end{displaymath}
Si noti che per $\alpha=1$ la distribuzione gamma coincide con l'esponenziale. Troviamo ora la generatrice dei momenti, la deriviamo e calcoliamo in 0 per ottenere:
\begin{gather*}
    E[X] = \phi'(0) = \frac{\alpha}{\lambda}\\
    Var(X) = E[X^2]-E[X]^2 = \frac{\alpha}{\lambda^2}
\end{gather*}
Se abbiamo più V.A. indipendenti di tipo gamma $X_i$, allora $\sum_i^n X_i$ è una gamma di parametri $(\sum_i^n \alpha_i,\lambda)$. Di conseguenza, se abbiamo una somma di esponenziali, avremo una gamma di parametri $(n, \lambda)$.
Se $Z_1, Z_2, ..., Z_n$ sono V.A. normali standard e indipendenti, allora la somma dei loro quadrati è una V.A. che prende il nome di \textbf{chi-quadro a n gradi di libertà}, e si indica con
\begin{displaymath}
    X \sim \chi_n^2
\end{displaymath}  
Essa è riproducibile, nel senso che se $X_1$ e $X_2$ sono due chi-quadro indipendenti, $X_1+X_2$ è una chi-quadro con $n_1+n_2$ gradi di libertà. Per dimostrare questo fatto non è necessario ricorrere alle funzioni generatrici, perché dalla definizione è evidente che $X_1+X_2$ è la somma dei quadrati di $n_1+n_2$ normali srandard indipendenti. 
Definiamo la quantità $\chi_{\alpha,n}^2$, con $\alpha$ reale tra 0 e 1:
\begin{displaymath}
    P(X\ge \chi_{\alpha,n}^2) = \alpha
\end{displaymath}
\subsubsection{La relazione tra chi-quadro e gamma}
Una chi-quadro ad n gradi di libertà, corrisponde a una gamma di parametri $(n/2, 1/2)$. La densità di probabilità è perciò data da:
\begin{displaymath}
    f(x) = \frac{x^{n/2-1}e^{-x/2}}{2^{n/2}\Gamma(n/2)}
\end{displaymath}
\subsection{Le distribuzioni T}
Se $Z$ e $C_n$ sono V.A. indipendenti, la prima normale std e la seconda chi-quadro con $n$ gradi di libertà, allora la V.A. $T_n$, definita come 
\begin{displaymath}
    T_n := \frac{Z}{\sqrt{C_n/n}}
\end{displaymath}
si dice avere distribuzione t con n gradi di libertà, denotata con
\begin{displaymath}
    T_n \sim t_n
\end{displaymath}
La densità delle distribuzioni $t$ è simmetrica rispetto all'asse di ascissa 0. È possibile dimostrare che al crescere di $n$, la densità di $t_n$ converge a quella della normale standard. 
Come sempre, calcoliamo valore atteso e varianza:
\begin{gather*}
    E[T_n] = 0\\ 
    Var(T_n) = \frac{n}{n-2}
\end{gather*}
Si noti che, al crescere di $n$, la varianza decresce, convergendo a 1, cioè alla varianza della n.s. In analogia con quanto fatto per le chi-quadro, definiamo $t_{\alpha,n}$
\begin{displaymath}
    P(T_n\ge t_{\alpha,n}) = \alpha
\end{displaymath}
Dalla simmetria rispetto allo zero otteniamo che
\begin{displaymath}
    -t_{\alpha,n} = t_{1-\alpha,n}
\end{displaymath}
\subsection{Le distribuzioni F}
Se $C_n$ e $C:m$ sono V.A. indipendenti, di tipo chi-quadro con $n$ e $m$ gradi di libertà, allora la V.A. $F_{n,m}$ definita da:
\begin{displaymath}
    F_{n,m} :=\frac{C_n/n}{C_m/m}
\end{displaymath}
si dice avere distribuzione F con $n$ e $m$ gradi di libertà. Come sempre, definiamo
\begin{displaymath}
    P(F_{n,m}>P_{\alpha,n,m}) = \alpha
\end{displaymath}
Notiamo, con calcoli, che
\begin{displaymath}
    \frac{1}{F_{\alpha,n,m}}=F_{1-\alpha,n,m}
\end{displaymath}
\subsection{Distribuzione logistica}
Una V.A. continua si dice avere distribuzione \textbf{logistica} di parametri $(\mu, \nu)$, con $\nu>0$, se la sua funzione di ripartizione è data da
\begin{displaymath}
    F(x) = \frac{e^{(x-\mu)/\nu}}{1+e^{(x-\mu)/\nu}}\hspace{10px}\forall x \in \mathbb{R}
\end{displaymath}
\textit{Bella eh?} Cerchiamo il valore atteso, e scopriamo che $E[X] = \mu$. $\nu$ è invece detta \textit{dispersione}. Una V.A. logistica con media 0 e dispersione 1 è detta logistica standard.
\end{document}
